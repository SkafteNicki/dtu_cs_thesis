\chapter{Examples of figures, tables, equations and listings}
In the following a bunch of examples of figures and tables have been made.

Some tips if you want good looking diagrams or graphs which will be inserted as pictures (e.g. in a figure environment with \textbackslash includegraphics): The main font is Arial. Use DTU colours as described in \cref{sec:colours}. Use high quality pictures. Try to scale the diagram (picture) so the text size of the axis legends match the text size in this document.

Remember to change the label of your figures so there are no duplicate labels. A label should be placed below a caption or after a heading (fx after a \textbackslash chapter).

\section{Tables and figures}

\begin{table}[H]
\centering
\caption{This is a \texttt{booktabs} table. Go to \url{http://www.tablesgenerator.com/} and use the booktabs table style}
\label{tab:tableExample}
\begin{tabular}{@{}llS@{}}
\toprule
\multicolumn{2}{c}{Item} &            \\ \cmidrule(r){1-2}
Animal     & Description & Price (\$) \\ \midrule
Gnat       & per gram    & 13.65      \\
           & each        & 0.01       \\
Gnu        & stuffed     & 92.50      \\
Emu        & stuffed     & 33.33      \\
Armadillo  & frozen      & 8.99       \\ \bottomrule
\end{tabular}
\end{table}
\texttt{Booktabs} tables don't use any vertical lines. Only horizontal lines are used. \Cref{tab:tableExample} begins with a \textbackslash \texttt{toprule}, ends with a \textbackslash \texttt{bottomrule} with \textbackslash \texttt{midrule} in between. The table has 3 columns formatted as \texttt{@\{\}llS@\{\}}. \texttt{@\{\}} is cropping the horizontal lines of the table to fit the content (removes column spacing at the left and right edges). \texttt{l} aligns the column to the left and \texttt{S} aligns the column according to the decimal point (\texttt{siunitx} package). You can of course also use \texttt{r} to align right or \texttt{c} to center the contents of the column.

\begin{table}[H]
\centering
\caption{Wrongly formatted table}
\label{tab:tableExampleWrong}
\begin{tabular}{llll}
\toprule
                    & Voltage & Current   & Power   \\
                    & V       & A         & W       \\ \midrule
Transformer input   & 234.4   & 0.50      & 117.4   \\ \midrule
Transformer output  & 25.86   & 2.72      & 70.3    \\ \midrule
Efficiency          &         &           & 60\%    \\ \bottomrule
\end{tabular}
\end{table}

\begin{table}[H]
\centering
\caption{Correctly formatted table}
\label{tab:tableExampleCorrect}
\begin{tabular}{@{}lSSS@{}}
\toprule
                    & {Voltage} & {Current} & {Power}       \\
                    & V         & A         & W             \\ \midrule
Transformer input   & 234.4     & 0.50      & 117.4         \\
Transformer output  & 25.86     & 2.72      & 70.3          \\ \midrule
Efficiency          &           &           & \SI{60}{\percent} \\ \bottomrule
\end{tabular}
\end{table}

\Cref{tab:tableExampleWrong} and \cref{tab:tableExampleCorrect} have the same comtents but there are some subtle differences in formatting which makes \cref{tab:tableExampleCorrect} the superior table of the two. The most obvious change is removing the midrule between the transformer input and output rows. The efficiency row is the odd man out and a midrule has been used to emphasise the difference between the transformer rows and the efficiency row. The delimiters in the voltage, current and power columns are aligned. The horizontal lines (rules) fits to the content and instead of protruding. The spacing between 60 and the percentage sign is correctly adjusted.

\begin{figure}[H]
\centering
\includegraphics[width=0.3\textwidth]{Pictures/Logos/dtured_cmyk.pdf}
\caption{Just a normal figure}
\label{fig:figure}
\end{figure}



\begin{figure}[H]
\centering
\begin{subfigure}{.5\textwidth}
  \centering
  \includegraphics[width=.4\linewidth]{Pictures/Logos/dtured_cmyk.pdf}
  \caption{A subfigure}
  \label{fig:twosub1}
\end{subfigure}%
\begin{subfigure}{.5\textwidth}
  \centering
  \includegraphics[width=.4\linewidth]{Pictures/Logos/black_cmyk.pdf}
  \caption{A subfigure}
  \label{fig:twosub2}
\end{subfigure}
\caption{A figure with two subfigures}
\label{fig:twosubfigures}
\end{figure}



\begin{figure}[H]
\centering
\begin{subfigure}{.49\textwidth}
  \centering
  \includegraphics[width=.3\linewidth]{Pictures/Logos/dtured_cmyk.pdf}
  \caption{A subfigure}
  \label{fig:foursub1}
\end{subfigure}%
\begin{subfigure}{.49\textwidth}
  \centering
  \includegraphics[width=.3\linewidth]{Pictures/Logos/black_cmyk.pdf}
  \caption{A subfigure}
  \label{fig:foursub2}
\end{subfigure}
\begin{subfigure}{.49\textwidth}
  \centering
  \includegraphics[width=.3\linewidth]{Pictures/Logos/black_cmyk.pdf}
  \caption{A subfigure}
  \label{fig:foursub3}
\end{subfigure}
\begin{subfigure}{.49\textwidth}
  \centering
  \includegraphics[width=.3\linewidth]{Pictures/Logos/dtured_cmyk.pdf}
  \caption{A subfigure}
  \label{fig:foursub4}
\end{subfigure}
\caption{A figure with four subfigures}
\label{fig:foursubfigures}
\end{figure}

Referring to the figure as a whole \cref{fig:foursubfigures} or to an individual sub figure \cref{fig:foursub1} is done the normal way with \texttt{\textbackslash cref\{\}} commands.


\section{Equations}
In-line math is easy. Anything surrounded by dollar signs becomes a math field. Here is an example: $f(x)=2x-1$. Also anything inside the ``\textbackslash begin\{equation\}'' and  ``\textbackslash end\{equation\}'' environment is also a math field. Examples are shown below.

All equations use the default latex font. Some might say it looks weird with a serif font for equations and a sans-serif font for the body text. However, it is very unpractical to change the math font in latex which is the exactly the reason why this has not been done. One benefit of the serif style math font is the clear distinction between symbols (variables) and units.

On the subject of units, those are all taken care of by the \texttt{\textbackslash siunitx} package. Whenever there is a number followed by a unit one should write \textbackslash SI\{number\}\{unit\}. Note this command is case sensitive. If a unit should follow a variable use the command \textbackslash si\{unit\} (also case sensitive).

The ideal gas law is shown in \cref{eq:idealgaslaw}.
\begin{equation} \label{eq:idealgaslaw}
    p \cdot V = n \cdot R \cdot T
\end{equation}

\begin{equation} \label{eq:IME}
    \frac{\partial}{\partial t} \int_{0}^{\delta} U dy = - \delta \frac{1}{\rho}\frac{\partial P}{\partial x}-U_f(t)^2
\end{equation}

\begin{equation} \label{eq:penDepthStep}
d_{step} = \sqrt{\frac{\delta}{\frac{dw}{dp_v}} \cdot t} =
\sqrt{\frac{\SI{1.0e-11}{kg/(m.s.Pa)}}{\frac{\SI{5.4}{kg/m^3}}{\SI{233.82}{Pa}}} \cdot \SI{7200}{s}} =
\SI{0.001766}{m} = \SI{1.766}{mm}
\end{equation}

\begin{equation*} % Equation without number
    x = \mathtt{x}, \mathbf{x}, \mathit{x}, x_{1_{2_{3_{4}}}}^{1^{2^{3^{4}}}} \cdot hello * \text{hello world} \cdot \text{equation without number}
\end{equation*}

Notice how the \texttt{aligned} environment can be used to align the equilibrium arrows in \cref{eq:equilibrium}. Only one equation number is generated using this method. Alternatively if you want an equation number for each line see \crefrange{eq:align1}{eq:align2}.

\begin{equation} \label{eq:equilibrium}
\begin{aligned}
    CH_3COOH + OH^{-} &\rightleftharpoons CH_3COO^{-} + H_2O \\
    H_2O &\rightleftharpoons H^{+}_{(aq)} + OH^{-}_{(aq)}
\end{aligned}
\end{equation}


\begin{align}
    \label{eq:align1}
    f(x) &= 1 + x - 3 x^2 \\
    \label{eq:align2}
    g(x) + y &= 3x - \frac{1}{2} x^3
\end{align}




\section{Listings (code)}

\Cref{lst:montecarlo} is a nicely formatted block of code. A listing will automatically continue on the next page if it encounters a page break. Many different programming languages can be highlighted. Check the \texttt{listings} package documentation for a list of supported programming languages.

\begin{lstlisting}[language=Matlab, caption = Monte Carlo simulation to estimate the value of $\pi$, label=lst:montecarlo]
%% Monte Carlo simulation, estimation of pi
m=1E7;

x=rand(m,1);
y=rand(m,1);

g = x.^2+y.^2-1;

%dots outside
Pf = sum((g)<=0)/m

pi = 4*Pf
\end{lstlisting}
